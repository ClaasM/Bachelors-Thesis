% !TEX root = ../main.tex
\chapter{Conclusion}
\label{ch:conclusion}


In this thesis, sentiment analysis and topic modeling were combined and applied to streamed social network data from Twitter.\\
It was theorized that organizations do not use social media to its full potential,
and that monitoring what users talk about, and how they feel about it,
could be of value,
while it was established that manual analyses would be too slow and labor-intensive.\\
Therefore, the goal was set to provide an automated real-time monitoring solution to semi-technical users.

In the following chapters, technologies and methods that could be used to accomplish this were explored and compared.
Kafka was introduced for streaming, and Spark to enable distributed analyses of incoming statuses from Twitter.
It was discovered that while Spark itself does not yet offer sufficient functionality to perform all the analyses,
it is still useful for this application since it can be used in conjunction with other libraries.\\
Still, for this scenario, the Kafka-Spark pipeline proved to be over-the-top.
However, in a multi-user scenario, with many more incoming statuses to be analyzed,
its scaling capabilities would make it worth the overhead.

While comparing sentiment analysis methods, it was found that sophisticated sentiment analysis methods do not adapt well
to the case of short Twitter statuses, and that a simple multinomial naive bayes on preprocessed text achieves the
best performance.
For example, tagging negated words had no effect, even though it improved results significantly when working with
other types of text artifacts~\cite{Hoffmann2005}.
The dataset used also proved to be highly topical, putting the models' validity into question,
even though it was already established to be the most suitable dataset.\\
For topic modeling, LDA and NMF were compared.
However, since the results were subjectively very similar, the choice fell on the more established LDA.\\
Both analyses were combined, to give an assessment of sentiment by topic, and the results were visualized,
indicating inherent differences in sentiment for different topics.

Sentiment analysis, topic modeling and the combination of both was then implemented as a user-friendly dashboard that would
allow users to monitor and analyse Twitter streams in real time without requiring programming abilities.
A set of challenges regarding the performance requirements when having to perform these analyses in real-time,
and the visualization of these results, were encountered and solved.\\
The resulting application showed the static topic model to not be perfectly accurate anymore,
due to only temporarily relevant topics.
Still, the application provides interesting insights into activity on Twitter,
and fulfills its purpose of enabling monitoring and analyses of streamed social network data.

The statement that Twitter with its API offers a never-seen-before opportunity for
many kinds of researchers~\cite{Kwak2010} could be confirmed,
and not only will the findings of this thesis be of value to other computer scientists,
but it is also a first step towards opening up this opportunity to a broader audience.

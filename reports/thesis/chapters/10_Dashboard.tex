% !TEX root = ../main.tex
\chapter{Dashboard}

\label{ch:dashboard}

A user-friendly streaming dashboard was created to visualize and monitor.
A different visualization was needed
This dashboard is proper overkill lol
This dashboard is designed for semi-tech users.
This dashboard only does classification, the models are trained beforehand with Notebooks and simple scripts.
This dashboard puts all the aforementioned theory into practice with streaming, making it one of the most important chapters.

\section{Requirement Engineering}
\label{sec:requirementEngineering}

Not only establishing the requirements of the dashboard, but also why a dashboard makes sense in the first place

\section{Architecture}
\label{sec:architecture}

Simple architecture diagram (MVC, client-server)

\section{Project Structure}
\label{sec:projectStructure}

The whole dashboard is under visualization/dashboard

Component-based project structure (vs. asset-based)

\section{Implementation}
\label{sec:implementation}

To combine topic modeling and sentiment analysis, they were all put into one sparkjob, to limit required bandwith to execution nodes (the dictionary doesn't need to be transmitted as often).
Also of course the amount of events emitted via the sockets is reduced.
It's okay since the jobs are still small enough for kafka to distribute them efficiently in a distributed setting.
And this makes it easier to combine the results of both analyses on a per-status (per-RDD) basis.

As explained in~\ref{subsec:nltk}, the NLTK is meant for teaching and research,
and the naive bayes classifier built with it did not meet the performance demands in a streaming scenario,
so scikit was used

\subsection{Backend}
\label{subsec:backend}

Flask was used because it is light-weight and unopinionated.

\subsection{Frontend}
\label{subsec:frontend}

Maybe a small charting/data visualization library comparison here?
This chapter is also around the 5 page mark in length.
\pagebreak[5]
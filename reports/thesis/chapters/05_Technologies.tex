% !TEX root = ../main.tex

\chapter{Technologies used}
\label{ch:technologiesUsed}

\paragraph
I introduce all the technologies used here
so that I can give code examples and talk about implementation
while introducing the theory for sentiment analysis and topic modeling.

\paragraph
I limit myself to technologies used across the whole project and introduce specific libraries like Gensim for LDA when they are used.

\paragraph
I exclude the dashboard here, or should I include stuff like Angular, Flask or HTML as technologies here?

\section{Languages}
\label{sec:languages}

Only Python

\section{Libraries}
\label{sec:libraries}

\subsection{Spark}
\label{subsec:spark}

Explaining Spark and how it was used
\begin{itemize}
    \item
    Spark offers some good data analysis functions (maybe show wordcount code here)
    \item
    Comparing the DataFrame-based and of the RDD-based API of Spark
    \item
    But Spark's real strength lies in the fact any library can be used on the execution node, which is why I use other libraries quite liberally for the best workflow (See chapters \ref{sentimentanalysis} and \ref{ch:topicModeling})
    \item
    The Factory pattern was used often because of how python handles scopes
    \item
    Why it makes sense and what is transmitted to the exec nodes in the end
    \item
    First using a queueStream, but it doesn’t work in the python API, switched to starting a Kafka stream (Kafka just has to be running somewhere, on connect the stream is connected to the Kafka broker and the sparkjob is immediately submitted with the session id as a parameter)
\end{itemize}

\subsection{NLTK}
\label{subsec:nltk}

Explaining the NLTK and how it was used, but not yet talking about the sentiment analysis algorithms it offers.
Those are introduced and explained in detail in their chapters.

\section{Project Structure}
\label{sec:projectStructure}


Using the DS cookie cutter, except the streaming folder is one under source because it is used by data to collect training data and by visualization for the dashboard.
Generally a modular approach that can be easily expanded with further analysis/monitoring beyond the LDA/Sentiment Analysis with Naive Bayes combination has been build.


\section{Architecture}
\label{sec:architecture}

Explaining the choice of architecture with some figures.
% TODO architecture diagram from the book
About 2 pages total introducing the technologies.

\pagebreak[2]
% !TEX root = ../main.tex

\chapter{Technologies used}

\label{ch:technologiesUsed}

I explain all the technologies used (excluding the dashboard) here, so that I can give code examples and talk about implementation
while introducing the theory for sentiment analysis and topic modeling, for the reader to be able to make the connection easier.

\textbf{Since it has great influence in the general structur: Do you think this is a good idea?}

The Libraries used for each of the algorithms will be explained in their respective chapters;
I only talk about technologes/design patterns etc that are used across all of them

NLTK is only really used in sentiment analysis, but since its such a prominent library, I'll do a general introduction here.
Is that okay, even though I don't introduce gensim here (since gensim is purpose-build for Topic Modeling)

Using the DS cookie cutter, except the streaming folder is one under source because it is used by data to collect training data and by visualization for the dashboard.
Generally a modular approach that can be easily expanded with further analysis/monitoring beyond the LDA/Sentiment Analysis with Naive Bayes combination has been build.

\begin{itemize}
    \item
    Spark offers some good data analysis functions (maybe show wordcount code here)
    \item
    Comparing the DataFrame-based and of the RDD-based API of Spark
    \item
    But Spark's real strength lies in the fact any library can be used on the execution node, which is why I use other libraries quite liberally for the best workflow (See chapters \ref{sentimentanalysis} and \ref{ch:topicModeling})
    \item
    The Factory pattern was used often because of how python handles scopes
    \item
    Why it makes sense and what is transmitted to the exec nodes in the end
    \item
    First using a queueStream, but it doesn\’t work in the python API, switched to starting a Kafka stream (Kafka just has to be running somewhere, on connect the stream is connected to the Kafka broker and the sparkjob is immediately submitted with the session id as a parameter)


\end{itemize}

\section{Architecture}
\label{sec:architecture}

Using the architecture from Book with a graphic cotaining ...

About 2 pages introducing the technologies.

\pagebreak[2]
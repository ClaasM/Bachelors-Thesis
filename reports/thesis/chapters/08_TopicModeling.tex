% !TEX root = ../main.tex
\chapter{Topic Modeling}
\label{ch:topicModeling}


%TODO be more specific what is used where here
Also, two dictionaries of the preprocessed datasets were made using the Gensim-library,
which will be explained in more detail in~\ref{ch:topicModeling} where the library is primarily used.
Only words with a minimum frequency of 1 were included in the dictionary, which resulted in 3961 words.
The dictionaries will be used for topic modeling.

\section{Methods}
\label{sec:methods}

Similar to the chapter about sentiment analysis.

\subsection{LDA}
\label{subsec:lda}

Show results on training and test data of both datasets, maybe even with and without preprocessing (to prove yet again that the preprocessing made sense)

Explain the difficulty to assess the right number of stopwords.

Using stopword list to kind of solve this.%https://stackoverflow.com/questions/19130512/stopword-removal-with-nltk}
%(Porter et al)

Wordcloud of training data set post stopwords removal (and comparison to pre stopword and pre preprocessing wordclouds in \ref{sec:streamingSampleDataset} and \ref{sec:theSandersDataset})

Genism offers required functionality. % (See sources from link)%https://radimrehurek.com/gensim/wiki.html#latent-dirichlet-allocation)

Also doing topic modeling on the Sanders dataset for the sake of completeness.

\subsection{LSA}
\label{subsec:lsa}

\section{Results}
\label{sec:results}

% TODO Can't call it "Comparison" in parallel to sentiment analysis since topic models are not really objectively comparable.
% TODO Can't call it "Results" since there is already a chapter that is named that.

4 pages planned for this chapter.
\pagebreak[4]
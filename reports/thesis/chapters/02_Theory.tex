% !TEX root = ../main.tex

\chapter{Background Theory}

\label{ch:backgroundTheory}


%Social media is not used to its full potential by e.g. companies
Social media applications offer great opportunity for companies,
and while they are widely adopted and used,
companies struggle to achieve measurable business success with their social media work \cite{Culnan2015}.
\par
% There is reason to believe companies would benefit from being able to analyze what their customers talk about (topics) and how they feel about these topics (sentiment by topic)
This thesis theorizes that monitoring what users talk about,
and how they feel about what they talk about,
will enable stakeholders to get a more quantifiable and structured understanding of their target group.
\par
Culnan et al. \cite{Culnan2015} % TODO how to cite this better?
argue that identifying value metric is an element of effective social media implementation for companies.
This thesis theorizes that the topics users talk about, the sentiment there interactions have and the combination of these are such metrics.
\par
Furthermore, they argue that absorptive capacity (message and interaction processing) is one of the three main elements of social media implementation for companies.
This thesis theorizes that companies benefit from topic modeling, sentiment analysis and the combination of those being part of this processing.
\par
% What it could be used for, example use case
Not only could this real-time analysis be used to monitor consumer sentiment,
it could also be used to react to changes in sentiment among users in real-time to detect, prevent and counteract
drastic mood shifts (sometimes called shitstorms).
People responsible for managing social-media accounts with big influence may be exposed to and fear these kinds of online communication-crisis,
and being tasked with managing such an event poses great difficulty \cite{Salzborn2015}.
\par
% Using it as a feedback cycle for A/B testing.
Another benefit of monitoring the sentiment regarding a topic in real-time is that it could used for product development,
to compare different iterations of a product.
It could also be useful to rapidly iterate social media marketing strategies.
%TODO add some graphic here, and maybe divide into sub-chapters
%TODO also a double pagebreak between chapters is very unnecessary.
% !TEX root = ../main.tex
\chapter{Appendix}
\label{ch:appendix}

\section{Preprocessing and Tokenization Function Factories}
\label{sec:preprocessingAndTokenizationFunctionFactories}
The Python code of the preprocessing and tokenization functions,
in the factory pattern, can be seen in~\autoref{code:preprocessing_tokenization}.

\begin{figure}
    \caption{Implementation of corpus-creation.}
    \label{code:preprocessing_tokenization}
    % @formatter:off
    \begin{minted}{python}
def preprocessor():
    """
    Removes all #hashtags, @mentions and other commonly used special characters used directly in front of
    or behind valid words as well as URL's and then only keeps valid words.
    :param text: the text to be preprocessed
    :return: the preprocessed text
    """

    def _preprocess(text):
        # Remove url's
        text = re.sub(r"http\S+", "", text)
        # Remove all the other stuff
        return " ".join([word for word in re.split("[\s;,.#:-@!?'\"]", text) if word.isalpha()])

    return _preprocess


def tokenizer(remove_stopwords=True):
    """
    Tokenization function for LDA. Used for training _and_ during streaming
    :return:
    """
    regex_tokenizer = RegexpTokenizer(r'\w+')
    # nltk stopword list plus some miscellaneous terms with low informativeness
    stoplist = set(['amp', 'get', 'got', 'hey', 'hmm', 'hoo', 'hop', 'iep', 'let', 'ooo', 'par',
                    'pdt', 'pln', 'pst', 'wha', 'yep', 'yer', 'aest', 'didn', 'nzdt', 'via',
                    'one', 'com', 'new', 'like', 'great', 'make', 'top', 'awesome', 'best',
                    'good', 'wow', 'yes', 'say', 'yay', 'would', 'thanks', 'thank', 'going',
                    'new', 'use', 'should', 'could', 'best', 'really', 'see', 'want', 'nice',
                    'while', 'know', 'que', 'sur', 'con'] + nltk.corpus.stopwords.words("english"))

    def _tokenize(text):
        # Tokenize
        tokens = regex_tokenizer.tokenize(text.lower())
        # Remove words with length < 3
        tokens = [token for token in tokens if len(token) > 2]
        if remove_stopwords:
            # Remove stop words
            tokens = [token for token in tokens if token not in stoplist]
        return tokens

    return _tokenize
    \end{minted}
    % @formatter:on
\end{figure}

\newpage

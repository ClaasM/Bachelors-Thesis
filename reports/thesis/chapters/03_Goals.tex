% !TEX root = ../main.tex
\chapter{Goals}
\label{ch:goals}

\textbf{Explaining what can "provide" the value explained in the previous chapter}


The goal of this thesis is to develop a solution that enables its users to monitor,
research and understand what twitter users talk about and how they feel about it in real time,
visualizing and quantifying the influence of spread and diffusion of information.
% TODO maybe some figure from Kwag 2010 if this looks to empty
\\
% TODO ask Dietrich: I aim to provide x or this thesis aims to provide x or will provide x? (Welche Person, welche Zeitform?)
A set of technologies appropriate for this application will be determined.
A quantitative comparison of methods used for sentiment analysis and topic modeling will be made,
and the best method for this use case determined.
\\
That way, this thesis will provide a way to analyze sentiment of tweets and model their topics,
as well as the combination of those.
\\
The final solution will provide a way to create a stream of real-time tweets and the associated analysis,
which can be filtered according to the users preferences to analyze only the activity most relevant to them.
\\
Most importantly, this solution will enable people not sufficiently familiar with technologies required to
implement such a real-time analysis application to nevertheless research the topics and sentiment of real-time,
streamed social network data.
\\
Furthermore, the application will be  of a modular design, so that it can be easily expanded with different forms of analysis,
without requiring a full understanding of all underlying technologies.
\\
In the process of implementation,
contributions to less-used but still important packages will be made.
%TODO some sort of figure
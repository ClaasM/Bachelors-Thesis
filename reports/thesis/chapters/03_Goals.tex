% !TEX root = ../main.tex
\chapter{Goals}
\label{ch:goals}

The goal of this thesis is to develop a solution that enables its users to monitor,
research and understand what twitter users talk about and how they feel about it in real time,
visualizing and quantifying their activity on Twitter.
\\
A set of technologies appropriate for this application will be decided on and explained.
A quantitative comparison of methods used for sentiment analysis and topic modeling will be made,
exploring previous approaches and the special use case provided by Twitter.
The best method for this use case will then be determined.
\\
That way, this thesis will provide a way to analyze sentiment of statuses and model their topics,
as well as the combination of those.
\\
The final solution will provide a way to create a stream of statuses, analyzed in real time as they are posted on Twitter,
which can be filtered according to the users preferences to analyze only the activity most relevant to them.
\\
Most importantly, this solution will enable people not sufficiently familiar with technologies required to
implement such a real-time analysis application themselves to nevertheless research the topics and sentiment of real-time,
streamed social network data.
\\
The application will be  of a modular design, so that it can easily be expanded with different forms of analysis,
without requiring a full understanding of all the underlying technologies.
% !TEX root = ../main.tex

\chapter{Motivation}
\label{ch:motivation}

% A definition of social media in general
Social media is defined as an application with a set of characterizing features \cite{Ellison2008}:
\begin{enumerate}
    \item
    Creating a profile containing personal information such as location or age, as well as pictures and optionally other kinds of multimedia content
    \item
    Building a list of connected profiles facilitating relationships such as friendship
    \item
    Seeing and traversing lists of connections from your own and other profiles
\end{enumerate}

% Some examples of social media, including twitter, which I focus on in this thesis

The first application that can be classified as social media according to these characterisitics
was the web-based service SixDegrees.com, launched in 1997 \cite{Ellison2008}.
A timeline of this and other selected social media application launches can be seen in \ref{fig:timeline}.

\begin{figure}
    \caption{A timeline of selected social media launches.}
    \label{fig:timeline}
    \begin{chronology}[5]{1997}{2017}{\linewidth}
        \event{1997}{\small{SixDegrees.com}}
        %\event{1999}{LiveJournal}
        %\event{1999}{AsianAvenue}
        %\event{1999}{BlackPlanet}
        %\event{2000}{MiGente}
        %\event{2001}{CyWorld}
        %\event{2001}{Ryze}
        %\event{2001}{Fotolog}
        \event{\decimaldate{22}{3}{2002}}{\small{Friendster}}
        %\event{\decimaldate{5}{5}{2003}}{\small{LinkedIn}}
        \event{\decimaldate{1}{8}{2003}}{\small{MySpace}}
        %\event{2003}{\small{Xing}}
        \event{\decimaldate{4}{2}{2004}}{\small{Facebook}}
        %\event{\decimaldate{10}{2}{2004}}{\small{Flickr}}
        \event{\decimaldate{15}{7}{2006}}{\small{Twitter}}
        \event{\decimaldate{6}{10}{2010}}{\small{Instagram}}
        \event{\decimaldate{28}{6}{2011}}{\small{Google Plus}}
        \event{\decimaldate{24}{1}{2013}}{\small{Vine}}
    \end{chronology}
\end{figure}

Apart from the 3 main characterizing features, social media applications vary greatly in functionality.
Most offer messaging and blogging capabilities, while some even allow for more specific use cases like live video-streaming \cite{Ellison2008}.

\begin{table}
    \caption{A comparison of social media applications}
    \label{tab:comparison}
    \resizebox{\textwidth}{!}{%
    \begin{tabular}{lllll} %
        \toprule
        & & \multicolumn{2}{c}{Main Relationship Type} & \\
        \cmidrule{3-4}
        Name
        & Main Content Types
        & Name
        & Characteristics
        & Platforms
        \\
        \midrule
        Facebook
        & Text, images, videos, live-stream
        & Friendship
        & Two-sided, balanced
        & Web and App
        \\
        \midrule
        Twitter
        & Text (limited to 140 characters)
        & Following
        & One-sided, unbalanced
        & Web and App
        \\
        \midrule
        Instagram
        & Images
        & Following
        & One-sided, unbalanced
        & Web and App
        \\
        \midrule
        Google Plus
        & Text, images, videos
        & Friendship
        & Two-sided, balanced
        & Web and App
        \\
        \midrule
        Vine
        & Videos
        & Following
        & One-sided, unbalanced %unreciprocated
        & App
        \\
        \bottomrule
    \end{tabular}}
    % How to do a footnote from the iEEE conference template: \hline \multicolumn{4}{l}{$^{\mathrm{a}}$Sample of a Table footnote.}
    %TODO figure footnote: unbalanced* -> few accounts with most of the followers (aks dietrich how to properly do footnotes)
    %TODO google plus: being in the same circle
\end{table}

As seen in~\ref{tab:comparison}, not all social media applications are solely web-based services.
Some, such as Vine, do not even offer a web-based interface anymore.
Also, only few of the social media applications analyzed offer networking capabilities in the sense of two-sided relationship initiation,
meaning there is no focus on creating relationships online that are then transferred into the real world.
For these reasons, this thesis refrains form using the term social networking site (SNS), as commonly used in previous works~\cite{Ellison2008},
and instead uses the term social media application.
\par

% Social media is big and growing
The first indication of the growing importance and influence of social media can be found by looking at the growth numbers:
Facebook, for example, has more than 1 billion users to date, and Twitter's userbase grew by 1,382\% between February 2008 and February 2009~\cite{mcgiboney2009twitter} alone.
\par

% Social media influences elections...
It is estimated that during the 2016 U.S. election, american adults read and remembered on the order of one to several so called fake news;
untruths portrayed as news favoring a specific candidate.
Although no assessment can be made as to whether the influence of fake news was pivotal to the election,
evidence can be found supporting the claim that exposure to fake news influences a voters sentiment~\cite{Allcott2017}.
\par
% ...and companies
Furthermore, social media applications changed the way organisations, communities and individuals communicate.
Customers expect companies to listen to and engage with them on these channels - and past approaches to customer care are ill-suited for this new environment~\cite{Kietzmann2011}.
\par
Most of the recent growth in social media usage happened in a important target demographic for companies -
65\% of american adults were using social media in 2015, almost 10 times more than in 2005, when only 7\% did~\cite{Perrin2015}.
\par
Similar developments can be observed in the amount of goods and services sold via the internet:
The share of e-commerce in the total U.S. retail sales grew from 4.2\% in the first quarter of 2010 to more than double that, 8.5\%, by the second quarter of 2017, a mere 7 years later~\cite{statistaECommerceGrowth}.
This growth will continue and the change from traditional commerce to e-commerce
will further influence the way customers interact with companies.
\par
The growing importance and influence of social media for companies and individuals,
paired with the potential for misuse as demonstrated during the 2016 election,
necessitates improved real-time monitoring capabilities for the affected stakeholders.
\par
From an organizational standpoint, complaints, comments, questions and suggestions by the customer are not only unstructured,
but also "designed by the customer", meaning that the organization has little influence on the way customers choose to communicate with them.
Since many customers are already familiar with social media, it often becomes the medium of choice for communicating with companies.
However, no standardized rules exist for these kind of transactions, making reporting, analysis and knowledge management on the companies side difficult~\cite{Culnan2015}.
\par
This thesis aims to explore and implement means of monitoring topics of interaction on social media in real-time,
and combine this analysis with the associated sentiment, to provide informative insights for these stakeholders.

% First justification why we are using twitter TODO is this okay here?
Twitter's main content type as shown in~\ref{tab:comparison} consists of small, simple text entries, called tweets, with a maximum length of 140 characters,
which makes it popular among researchers, and a good fit for the proof of concept that this thesis aims to provide.
There are also technical benefits to using Twitter, which are explained further in~\ref{ch:twitter}.
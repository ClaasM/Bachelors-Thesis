% !TEX root = ../main.tex

\chapter{Structure}
\label{ch:structure}

\section{Recap}
\label{sec:recap}

% Giving a recap of what has already happened first
The first part of this thesis was to state the motivation,
establish a theory and the goals of this thesis, and introduce the technologies that will be used.
\\
A definition of social media was made,
some examples of social media were introduced and compared,
and the growth and its impact quantified.
\\
Afterwards, theories stating that social media is not used to its full potential by companies and other entities,
and that it provide more value with the right tools and knowledge, were developed. \\
More specifically, it was theorized that monitoring sentiment and topics in real-time could, for example,
be used to detect and react to drastic moodshifts called shitstorms, or as a feedback mechanism during product development.
\\
The goal of this thesis was set to be to create a solution that enables people with varying technical prowess to monitor
which topics social media users are talking about and how they feel about these topics in real-time.
\\
Technologies that will be used to achieve this task were introduced, among them the language of choice for this thesis,
Python, as well as frameworks and libraries that will be used for modeling, analysis and streaming like the NLTK or Spark,
and the project structure was established and explained.

\section{Outlook}
\label{sec:outlook}

After giving a general introduction, the thesis will now go into the specific use-case that will be focused.
\\
Since this thesis will focus on Twitter as a easily accessible and analyzable medium,
it will be introduced, evaluated and explored in-depth both from a user and a technological perspective in the next chapter.
Also, the datasets used in the~\ref{ch:sentimentAnalysis} and~\ref{ch:topicModeling} will be created and explained.
\\
Then, sentiment analysis will be introduced and applied to data from Twitter.
Several different methods of sentiment analysis will be applied to different datasets.
They will be evaluated, quantified and compared, and a decision on which method is best-suited for this use-case will be made.
\\
The same will then be done with topic modeling.
Also, a visualization will developed for topic models.
\\
As the most important part of this thesis, the two will then be combined and the results visualized and evaluated.
\\
To satisfy the requirement of enabling users from all disciplines to benefit from this combination,
a web-based dashboard is build that can be used to perform topic modeling and sentiment analysis on a
wide variety of easily configurable streams from Twitter, and that will visualize these and the combination of the two in an intuitive fashion.
\\
The value that combining sentiment analysis and topic modeling in this dashboard
will provide for the previously established stakeholders is then discussed.
Also, the research results of this thesis are reiterated, and a conclusion is drawn.
Lastly, opportunities for future work on the subject will be discussed.
